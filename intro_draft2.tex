\chapter{Introduction}
In today’s era of digitalization, it has become quite common and, to a certain extent, inevitable to use digital mediums, i.e., internet banking, digital transactions, communication, cryptocurrency, and even shopping. The extensive use of such mediums has opened doors for criminals who would illegally take advantage of these systems for their benefit due to their complexity and vastness. The systems in cyberspace are prone to flaws and backdoors, creating a potential flux of security risks that may cause enormous harm to individuals and entities. This area is where the field of cybersecurity plays a vital role. Cybersecurity describes a discipline specialized in protecting cyberspace’s various systems, using dedicated tools, design methods, analysis, policies, best practices, and testing techniques \citep[p.~22]{Anderson2020}. It is crucial to understand that this branch of knowledge deals with common vulnerabilities, i.e., Denial-of-Service (DOS) attacks, Malware, Phishing, SQL Injection, Man-in-the-Middle (MITM) attacks, and Cross-site scripting (XSS) attacks \citep[p.~3173]{Humayun2020}. Understanding these common attacks in detail is the key to mitigating and protecting an individual’s and organization’s assets from harm. Still, it will certainly not be enough for the countless unexpected events occurring in this domain. This is where well-trained and experienced cybersecurity professionals come into action.

\paragraph{}
Cyberattacks are increasing in size and weight all around the world, also at this very moment. Due to the sheer amount of systems and information in cyberspace, stealing and misusing one’s data brings more damage, and protecting it has gained more importance. For instance, in the United States, the damage caused by cyber-attacks has risen from 17.8 million to 6.9 billion US dollars between 2001 and 2021 \citep{Statista2022}. Similarly, in 2021, a small rural district in the eastern German state became the victim of ransomware. This cyber attack manages to get into one’s system and encrypt all the data using an encryption key, making the system unusable. This incident left the whole town without critical infrastructure for a few days, leaving people unable to use public services, i.e., welfare and address registrations \citep{DW2021}. Certainly, such incidents happen unexpectedly and instantaneously, which is hard to contain. However, the backup measures for such events were not carefully planned at an adequate level, and concrete pre-planning anticipating such attacks could have made this disaster less impactful or even prevented it entirely. Hence, we should emphasize securing the ever-evolving systems and generating awareness among the general public about cybersecurity. The need for professionals who understand and prevents disruptions are getting more relevant. Otherwise, we as individuals, organizations, and even as a nation will be destabilized, which is hard to recover quickly or, worse, not at all. 

\paragraph{}
The consistent aggression in digital space has left us vulnerable, and a significant number of cybersecurity experts is needed to strengthen the pillars of security. It is approximated that about 70\% of enterprises believe that the lack of cybersecurity expertise has caused harm to their organization, and more than 65\% are overwhelmed by the constant changing skill set required to keep up with the dynamic changes in this field \citep[section.~5]{CUT}. In addition to the lack of experts, there is also the human element, which is one of the easiest backdoors to get passed any system, and this is the cause of countless breaches \citep{McEvoy2019}. It is because there is a notable lack of knowledge about this topic. Simply sharing rudimentary information about what to avoid could make a big difference.

\paragraph{}
Thus, training and educating cybersecurity professionals is crucial to create safe and well-fortified cyberspace. The postgraduate degree in cybersecurity is a platform for individuals that provides the necessary skill set to understand and learn different cybersecurity techniques, i.e., threat detection, risk mitigation, application security, network security, and various possible cyber-attacks. Especially with a postgraduate degree in cybersecurity, one will be acquainted with the different tools and techniques of cybersecurity that can be applied professionally and be able to work as one of the on-demand experts in the field. Understanding this topic in detail would enable individuals to participate and actively support society by avoiding disruptions and harm in cyberspace.

\paragraph{}
On a personal level, one can use the skills gained from the postgraduate degree to stop or prevent cyber attacks and disasters and contribute to the community by providing valuable information to the public. Also, these tools and techniques will assist in preserving the integrity and confidence in using the internet/cyberspace for those who need them. Consequently, small changes can bring considerable differences to creating a stable environment when people start to understand what kind of threats they face.
