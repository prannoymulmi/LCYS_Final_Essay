\chapter{Cyber Security Technologies}
The threats identified in chapter \ref{chap:threat_modeling} cause severe disruptions if not mitigated properly. This chapter discusses two relevant technologies, SIEM and firewall, that could be applied in the context of ASMIS to further secure Queens medical center's critical infrastructure.

\section{Security Incident and Event Management}
SIEM is a cybersecurity technology that gathers logs from multiple network sources, i.e., routers, firewalls, servers, load balancers, and more. The tool provides transparency to the experts because it processes large amounts of log files, which would either be impossible or too cumbersome for a person to carry out. SIEM also efficiently analyzes the logs collected and notifies suspicious anomalies within the system \citep[p.~1]{SIEM}. For the clinic, adopting such instruments is of great value because it dramatically enhances the ability to respond to cyber incidents, improves traceability, and helps prevent previous mistakes. \newline\newline 
However, using the tool also has its downsides, and the first and foremost is the implementation time. The network architecture is usually complex. In simple terms, the organization needs several dedicated experts to install an effective and suitable SIEM system accommodating internal requirements. Additionally, if not configured properly, the system can produce many false positive results, resulting in people neglecting some notifications, leading to a significant threat.

\section{Firewall}
As the name suggests, a firewall is a barrier that sits in front of the network and filters packages that it deems safe into the network. The firewall provides an easy method to define which set of IP sources are allowed to go in and out of the web, and it is effortless to set up an Access Control List. This technology would be suitable also for the ASMIS staff system (See Figure. \ref{fig:dfd_staff}). The Staff system is more vulnerable to threats because gaining access to this system can disrupt the functionality of the critical infrastructure.\newline\newline
There are many different types of firewalls, i.e., packet filters, Circuit filters, and Web Application firewalls (WAF). Using the WAF, the staff could only access the system using the internal network, and any other network is filtered with SQL injection and XSS detection. Compared to simple packet filtering, WAF offers these extra advantages \citep[p.~354]{thang2020improving}. In addition, the clinic can set up a VPN with end-to-end encryption access for the staff, allowing them to use the system from anywhere.\newline\newline
Although firewalls seem secure, it also provides a false sense of protection because they are not very useful against internal threats and cannot stop social engineering, which can get them access to the network. A firewall is a staple technology used in network security but is limited only to filtering known impacts.