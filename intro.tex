\chapter{Introduction}
The healthcare sector is a vital part of our lives, and we made countless advancements in this field, improving our livelihood. The mere thought of not having accessible healthcare is catastrophic and can lead to countless loss of lives. Therefore, we must take great caution in improving communication between patients and healthcare providers to ensure everyone gets the proper required service.\newline \newline
Similarly, Queens Medical Center is one of many healthcare centers serving the 
residents as the first point of contact for any health-related issues. The new automated appointment system will provide a web interface between the clinic and the patient, where patients can easily book an appointment using a computer or a mobile device with access to the internet.\newline\newline
Deployment of the ASMIS system has numerous benefits, namely, saving human resources and time. The medical staff is already under large workloads, and with this system, they can redirect their energy to other sectors. The system also has positive effects on the user side, where patients do not have to wait for clinic personnel to pick up the phone; instead, they open the application and book an appointment instantly. In particular, automation drastically reduces human error in scheduling, which could sometimes be a matter of life and death. In the long run, digitalization will save more money and make the system more efficient.\newline\newline
Using the ASMIS system has its advantages but, on the contrary, is not free from the vulnerabilities that cyberspace brings along. For the application to work as intended, it needs to store the personal data of the patient, specialists, and other staff. Storing such personally identifiable information (PII) creates a significant risk to privacy \citep[p.~374]{IOT}. If data is stolen or leaked, crooks can use the data for illegal activities or sell it. In addition to data leaks, the danger is always lurking, which could disrupt the functionality of critical infrastructure. External threats, such as DDoS attacks, malware, viruses, and advanced persistent threats, are more everpresent. These attacks cause financial, reputational damage, and damage to lives \citep[p.~377]{IOT}.\newline\newline
The main aim of this report is to analyze different threats and levels of risks present in the system. Threat modeling and risk managing techniques will lead to a more robust and secure Appointment Scheduling Management Information System (ASMIS) and raise user confidence.